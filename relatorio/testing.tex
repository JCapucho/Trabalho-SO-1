\chapter{Testagem}

\section{Metodologia}\label{sec:testing_metodology}

Durante o desenvolvimento dos scripts em questão estes começaram por ser
testados manualmente, no caso do \Verb|spacecheck.sh| este era testado com alguns
ficheiros e diretórios criados a mão para simular diferentes casos relevantes ao
funcionamento do script. O \Verb|spacerate.sh| utilizava os ficheiros produzidos
por estes testes com algumas ligeiras modificações feitas à mão, para simular casos
mais extremos.

No entanto este processo era menos que ideal, pois não só obrigava a gastar
tempo a configurar manualmente os diretórios e a editar ficheiros do
\Verb|spacecheck.sh|, mas também a verificar que os
resultados correspondiam ao esperado. Mas o maior problema de todos era o erro
humano inerente a realização da testagem desta maneira, pois não só dependemos
do desenvolvedor lembrar-se de todos os casos que têm de testar, mas também que
estes sejam bem configurados de todas as vezes e que os erros que possam
eventualmente aparecer sejam apanhados.

Estes dois últimos pontos em particular são mais propícios de falhar com o
aumento da frequência com que são feitos \cite{fewster1999software, rafi2012benefits},
o que vai contra o nosso objetivo de testar frequentemente.

Logo procuramos obter um novo processo de testagem que satisfizesse os seguintes
objetivos:

\begin{enumdescript}
	\item[Automático]
	todos os testes necessários são executados sem que o desenvolvedor tenha que
	se lembrar de cada um deles individualmente.
	\item[Declarativo]
	todos os testes são configurados uma vez e sempre que os testes são
	executados os diretórios e ficheiros necessários são criados de raiz
	para garantir que os testes são facilmente reproduzidos.
	\item[Transparentes]
	não é necessário a intervenção do desenvolvedor para verificar o resultado
	do teste em casos em que este está correto.
\end{enumdescript}

Como tal desenvolvemos um processo automático de testagem que se baseia no
padrão de \textbf{Golden Testing}\cite{goldenTestsAreTasty}.
Este consiste em correr um teste com entradas
definidas e constantes para o mesmo teste (no nosso caso estas entradas são os
ficheiros e diretórios) e comparar o resultado com outro resultado prévio que se
sabe estar correto.

Esta abordagem é comum no desenvolvimento de \emph{User Interfaces}
\cite{ebayScreenshotTesting} e em mensagens de erros de ferramentas
\cite{brsonRustTested} pois em ambos destes casos a única propriedade\ que faz
sentido verificar é o resultado em si e a testagem de componentes individuais
destes sistemas não faz sentido em termos de prevenir erros.

\section{Implementação}

O novo processo de testagem para o \Verb|spacecheck.sh| e \Verb|spacerate.sh|
é implementado nos scripts \emph{Bash} \Verb|spacecheck-tests.sh| e
\Verb|spacerate-tests.sh|, respetivamente. O seu funcionamento de alto nível é
apresentado na figura \ref{fig:testing_diagram}.

\begin{figure}[H]
	\centering
	\begin{tikzpicture}[node distance=2cm]
		\node (start)
[flowchart_endpoint, text width=4cm, fill=green!40]
{\Verb|spacecheck-tests.sh|};

\node (source_tests)
[flowchart_step, below of=start]
{Carregar o arnês de testes};
\node (tests)
[flowchart_input, right of=source_tests, text width=3.5cm, xshift=3cm]
{\Verb|test_harness.sh|};

\node (filter_tests) [flowchart_step, below of=source_tests] {Filtrar testes};
\node (cli_args)
[flowchart_input, right of=filter_tests, text width=3.5cm, xshift=3cm]
{Argumentos de linha de comando};

\node (loop_tests)
[flowchart_decision, below of=filter_tests, yshift=-1cm]
{Testes por executar?};

\node (success)
[flowchart_endpoint, right of=loop_tests, xshift=3cm, fill=OliveGreen!40]
{Todos os testes passaram};
\node (run_test) [flowchart_step, below of=loop_tests, yshift=-1cm] {Executar um teste};

\node (diff)
[flowchart_decision, below of=run_test, yshift=-1cm]
{Diff com ficheiro de referência};
\node (failure)
[flowchart_endpoint, right of=diff, xshift=4.5cm]
{Testes falharam};

\draw [flowchart_arrow] (start) -- (source_tests);
\draw [flowchart_arrow] (tests) -- (source_tests);
\draw [flowchart_arrow] (source_tests) -- (filter_tests);
\draw [flowchart_arrow] (cli_args) -- (filter_tests);
\draw [flowchart_arrow] (filter_tests) -- (loop_tests);
\draw [flowchart_arrow] (loop_tests) -- node[anchor=south] {não} (success);
\draw [flowchart_arrow] (loop_tests) -- node[anchor=east] {sim} (run_test);
\draw [flowchart_arrow] (run_test) -- (diff);
\draw [flowchart_arrow] (diff) -- node[above] {Há diferenças} (failure);
\draw [flowchart_arrow]
(diff.west)
-- ++ (-1,0)
|- node[pos=0.25,anchor=east] {Não há diferenças} (loop_tests.west);

	\end{tikzpicture}
	\cprotect\caption{Diagrama do funcionamento de alto nível do \Verb|spacecheck-tests.sh|.}
	\label{fig:testing_diagram}
\end{figure}

\newpage

O script de testagem começa por definir funções que possam ser uteis a testagem
e de seguida define o mapa de testes. O mapa de testes é um array associativo
onde os índices são o nome dos testes e os valores são funções usadas para
configurar o teste (ou \Verb|:| que em \emph{Bash} é a função vazia).


\begin{listing}[H]
	\centering
	\begin{minted}{bash}
		quoting_test() {
			create_test_file "simple/file" 4
			create_test_file "with spaces/file" 2
			create_test_file 'with " quotes/file' 7
		}

		declare -A TESTS
		TESTS["empty_root"]=":"
		TESTS["quoting_test"]="quoting_test"
	\end{minted}
	\cprotect\caption{Exemplo de um mapa de testes e funções associadas.}
\end{listing}

Após o mapa de testes, é necessário definir duas funções extra \Verb|test_start|
e \Verb|test_end|, estas funções serão usadas pelo arnês de testes e vão correr
antes e depois de cada teste respetivamente. No \Verb|test_start| é suposto
definir código comum a todos os testes, como por exemplo, apagar ficheiros
antigos e criar o diretório do teste. O \Verb|test_end|, além de código definido
pelo desenvolvedor, têm de executar o script sobre teste e fazer o \emph{diff}
com os ficheiros de referência.

Finalmente o script de testagem carregá o ficheiro \Verb|test_harness.sh|. Este
ficheiro é o arnês de testagem, ou seja inclui rotinas de como fazer o
\emph{diff} com o ficheiro de referência e incluí o código que irá processar
os argumentos passados pelo utilizador e executar os testes.

O arnês foi desenhado para aceitar os nomes dos testes a correr como argumentos
na linha de comando, como é apresentado no código \ref{code:test_harness_filter}.

\begin{listing}[H]
	\centering
	\begin{minted}{shell-session}
		$ ./spacecheck-tests.sh empty_root quoting_test
		empty_root: PASSED
		quoting_test: PASSED
	\end{minted}
	\cprotect\caption{Exemplo de filtragem dos testes a executar.}
	\label{code:test_harness_filter}
\end{listing}

Esta facilidade foi implementada para possibilitar um ciclo de desenvolvimento
mais rápido quando o desenvolvedor está interessado num conjunto restrito de
testes, seja porque está a consertar o script ou porque está a escrever mais
testes.

Finalmente, o arnês vai executar todos os testes especificados (se nenhum foi
especificado, todos os testes definidos no mapa de testes serão executados).

\cprotect\subsection{\Verb|spacecheck.sh|}

Cada teste do \Verb|spacecheck.sh| começa por limpar
qualquer vestígios de execuções anteriores, de seguida o diretório do teste
é criado, isto tudo acontece no \Verb|test_start|.

De seguida cada teste cria vários ficheiros e diretórios de acordo com o
comportamento que se pretende testar, o código \ref{code:test_definition} mostra
um exemplo de um dos testes definidos.

\begin{listing}[H]
	\centering
	\begin{minted}{bash}
		quoting_test() {
			create_test_file "simple/file" 4
			create_test_file "with spaces/file" 2
			create_test_file 'with " quotes/file' 7
		}
	\end{minted}
	\cprotect\caption{Exemplo da definição de um teste do \Verb|spacecheck.sh|.}
	\label{code:test_definition}
\end{listing}

A função \Verb|create_test_file| é uma função auxiliar que cria um ficheiro num
dado caminho (e os diretórios onde ele reside) com um tamanho definido, a função
é apresentada na listagem de código \ref{code:create_test_file}.

\begin{listing}[H]
	\centering
	\begin{minted}{bash}
		create_test_file() {
			local path="$1"
			local size="$2"

			mkdir -p -- "$( dirname -- "$path" )"
			dd if=/dev/zero "of=$path" "bs=$size" count=1 2>/dev/null
		}
	\end{minted}
	\cprotect\caption{Definição da função \Verb|create_test_file|.}
	\label{code:create_test_file}
\end{listing}

A função faz uso do comando \bashinline|dirname| para obter o caminho sem o nome
do ficheiro, ou seja, apenas com os diretórios, de seguida passa este valor ao
\bashinline|mkdir| que os vai criar, aqui é utilizada a opção \Verb|-p| para
criar os diretórios recursivamente e para o \bashinline|mkdir| não falhar se
estes já existirem.

Em ambos estes comandos foi necessário utilizar \Verb|--| antes do caminho, isto
acontece porque é perfeitamente válido o caminho começar por um hífen
(\Verb|-|), que normalmente denota opções, isto foi explicado em maior detalhe
na secção \ref{sec:implementation_find_dirs}.

O ficheiro em si é criado através do comando \bashinline|dd|, este é utilizado
porque permite o controlo preciso do tamanho do ficheiro, para isto é utilizado
o ficheiro especial \Verb|/dev/zero| (um ficheiro sem fim que retorna sempre 0),
de onde vai ser lido um bloco (\Verb|count=1|) de \Verb|size| bytes
(\Verb|bs=$size|) que será copiado para o novo ficheiro.

\subsubsection{Problemas com variavéis externas}

Este tipo de testagem assume que o resultado da execução é \emph{determinista},
ou seja, para conjuntos de entradas iguais o resultado será igual, no entanto
o \Verb|spacecheck.sh| não pode ser considerado determinista pois este depende
de uma variável externa, o tempo.

O resultado de \Verb|spacecheck.sh| inclui sempre uma data \Verb|YYYYMMDD| de
quando este foi executado, isto é um problema porque o mesmo teste quando
executado em dias diferentes irá apresentar resultados diferentes causando com
que o teste falhe.

Logo para obter determinismo na execução dos testes é necessário executar o
script sempre com um tempo pré-definido, para tal foi utilizado o programa
\Verb|faketime| do projeto \Verb|libfaketime|
\footnote{\url{https://github.com/wolfcw/libfaketime}},
este permite mudar o tempo reportado a um processo sem ter de mudar o relógio
do sistema, sendo assim ideal para o nosso objetivo de uma ferramenta de fácil
utilização para testagem.

\begin{listing}[H]
	\centering
	\begin{minted}{shell-session}
		$ date +%Y%m%d
		20231104
		$ faketime "2023/11/01" date +%Y%m%d
		20231101
	\end{minted}
	\cprotect\caption{Exemplo do funcionamento de \Verb|faketime|.}
\end{listing}

\cprotect\subsection{\Verb|spacerate.sh|}

O \Verb|spacerate.sh| é testado de uma maneira semelhante ao
\Verb|spacecheck.sh|, no entanto agora invés de serem criados os ficheiros e
diretórios para cada teste, os ficheiros para comparar já existem e foram
desenhadas especialmente para manifestar comportamentos específicos no script.

Além disso o \Verb|spacerate.sh| ainda difere em mais um aspeto, o teste não é
executado apenas uma vez, mas sim \textbf{duas} vezes, a segunda vez com a ordem
dos ficheiros trocada. Como a troca da ordem dos ficheiros deve produzir um
resultado simétrico, a testagem nas duas direções é uma maneira simples de
verificar esta propriedade no script.

\section{Resultados}

A implementação do processo de testagem automática descrito nas secções
anteriores, permitiu-nos satisfazer os objetivos que delineamos na secção
\ref{sec:testing_metodology}, os testes são todos executados automaticamente
(com a opção de opcionalmente restringir os testes a executar), eles correm em
ambientes que são criados de raiz em cada execução e são declarados num ficheiro
\emph{Bash}.

Mas o mais importante de tudo, foi o facto do envolvimento do desenvolvedor
em todo o processo ter sido cortado significativamente, este agora só necessita
de começar o processo de testagem, verificar os ficheiros de referência
inicialmente e corrigir erros que sejam capturados pela testagem.

Isto traduz-se numa redução do tempo de testagem, que antes para
\textbf{um teste} era de \numrange{10}{30} segundos em média, para
$< 2$ segundos para \textbf{todos os testes!}

Além da redução do tempo necessário para a testagem, o novo processo foi logo
após a sua implementação capaz de capturar um erro que tinha sido introduzido
anteriormente aquando de mudanças de como os ficheiros eram filtrados no
\Verb|spacecheck.sh|. O erro causava com que o filtro de data de modificação não
tivesse qualquer efeito, este em particular é um dos casos mais difíceis de
testar devido ao números de passos necessários para configurar um diretório com
datas especificas manualmente.

Concluindo, o novo processo de testagem é mais eficiente e eficaz em garantir
a qualidade do trabalho criado, consideramos assim que o tempo gasto na sua
implementação foi útil e recuperado no longo termo.
