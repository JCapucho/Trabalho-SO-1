\chapter{Testagem}

\section{Metodologia}\label{sec:testing_metodology}

Durante o desenvolvimento do programa \Verb|spacecheck.sh| este foi testado
manualmente com alguns ficheiros e diretórios criados a mão para simular
diferentes casos relevantes ao funcionamento do programa.

No entanto este processo era menos que ideal, pois não só obrigava a gastar
tempo a configurar manualmente os diretórios mas também a verificar que os
resultados correspondiam ao esperado. Mas o maior problema de todos era o erro
humano inerente a realização da testagem desta maneira, pois não só dependemos
do desenvolvedor lembrar-se de todos os casos que têm de testar, mas também que
estes sejam bem configurados de todas as vezes e que os erros que possam
eventualmente aparecer sejam apanhados.

Estes dois últimos pontos em particular são mais propícios de falhar com o
aumento da frequência com que são feitos, o que vai contra o nosso objetivo
de testar frequentemente.

Logo procuramos obter um novo processo de testagem que satisfizesse os seguintes
objetivos:

\begin{enumdescript}
	\item[Automático]
	todos os testes necessários são executados sem que o desenvolvedor tenha que
	se lembrar de cada um deles individualmente.
	\item[Declarativo]
	todos os testes são configurados uma vez e sempre que os testes são
	executados os diretórios e ficheiros necessários são criados de raíz
	para garantir que os testes são facilmente reproduzidos.
	\item[Transparentes]
	não é necessário a intervenção do desenvolvedor para verificar o resultado
	do teste em casos em que este está correto.
\end{enumdescript}

Como tal desenvolvemos um processo automático de testagem que se baseia no
padrão de \textbf{Golden Testing}. Este consiste em correr um teste com entradas
definidas e constantes para o mesmo teste (no nosso caso estas entradas são os
ficheiros e diretórios) e comparar o resultado com um resultado prévio que se
sabe estar correto.

Esta abordagem é comum no desenvolvimento de \emph{User Interfaces} e em
mensagens de erros de ferramentas pois em ambos destes casos a única propriedade
que faz sentido verificar é o resultado em si e a testagem de componentes
individuais destes sistemas não faz sentido em termos de prevenir erros.

\section{Implementação}

O novo processo de testagem é implementado no script \emph{Bash}
\Verb|test_harness.sh|, o seu funcionamento de alto nível é descrito na
figura \ref{fig:testing_diagram}.

\begin{figure}[H]
	\centering
	\begin{tikzpicture}[node distance=2cm]
		\node (start)
[flowchart_endpoint, text width=4cm, fill=green!40]
{\Verb|spacecheck-tests.sh|};

\node (source_tests)
[flowchart_step, below of=start]
{Carregar o arnês de testes};
\node (tests)
[flowchart_input, right of=source_tests, text width=3.5cm, xshift=3cm]
{\Verb|test_harness.sh|};

\node (filter_tests) [flowchart_step, below of=source_tests] {Filtrar testes};
\node (cli_args)
[flowchart_input, right of=filter_tests, text width=3.5cm, xshift=3cm]
{Argumentos de linha de comando};

\node (loop_tests)
[flowchart_decision, below of=filter_tests, yshift=-1cm]
{Testes por executar?};

\node (success)
[flowchart_endpoint, right of=loop_tests, xshift=3cm, fill=OliveGreen!40]
{Todos os testes passaram};
\node (run_test) [flowchart_step, below of=loop_tests, yshift=-1cm] {Executar um teste};

\node (diff)
[flowchart_decision, below of=run_test, yshift=-1cm]
{Diff com ficheiro de referência};
\node (failure)
[flowchart_endpoint, right of=diff, xshift=4.5cm]
{Testes falharam};

\draw [flowchart_arrow] (start) -- (source_tests);
\draw [flowchart_arrow] (tests) -- (source_tests);
\draw [flowchart_arrow] (source_tests) -- (filter_tests);
\draw [flowchart_arrow] (cli_args) -- (filter_tests);
\draw [flowchart_arrow] (filter_tests) -- (loop_tests);
\draw [flowchart_arrow] (loop_tests) -- node[anchor=south] {não} (success);
\draw [flowchart_arrow] (loop_tests) -- node[anchor=east] {sim} (run_test);
\draw [flowchart_arrow] (run_test) -- (diff);
\draw [flowchart_arrow] (diff) -- node[above] {Há diferenças} (failure);
\draw [flowchart_arrow]
(diff.west)
-- ++ (-1,0)
|- node[pos=0.25,anchor=east] {Não há diferenças} (loop_tests.west);

	\end{tikzpicture}
	\cprotect\caption{Diagrama do funcionamento de alto nível do \Verb|test_harness.sh|.}
	\label{fig:testing_diagram}
\end{figure}

O script começa por carregar outro ficheiro \emph{Bash} \Verb|tests.sh| onde os testes
estão definidos, isto é feito por uma questão de organização,
\Verb|test_harness| define o processo de testagem e funções de utilidade para os
testes enquanto que \Verb|tests.sh| contêm os testes.

O script foi desenhado para aceitar os nomes dos testes a correr como argumentos
na linha de comando, como é apresentado no código \ref{code:test_harness_filter}.

\begin{listing}[H]
	\begin{minted}{shell-session}
		$ ./test_harness.sh empty_root quoting_test
		empty_root: PASSED
		quoting_test: PASSED
	\end{minted}
	\cprotect\caption{Exemplo de uso do \Verb|test_harness.sh| com filtragem.}
	\label{code:test_harness_filter}
\end{listing}

Esta facilidade foi implementada para possibilitar um ciclo de desenvolvimento
mais rápido quando o desenvolvedor está interessado num conjunto restrito de
testes, seja porque está a consertar o programa ou porque está a escrever mais
testes.

Após fazer esta filtragem, este vai executar todos os testes especificados
(ou todos se nenhum teste for especificado). Cada teste começa por limpar
qualquer vestígios de execuções anterior, de seguida a diretoria do teste
é criada e preparada de acordo com o teste, este pode ainda especificar
argumentos a passar ao \Verb|spacecheck.sh|.

O resultado da execução é comparado ao ficheiro de referência do teste usando o
comando \Verb|diff|, caso o ficheiro de referência não exista esta execução é
assumida como o ficheiro de referência para execuções futuras (cabendo ao
desenvolvedor verificar que o resultado é o esperado). Se o resultado diferir,
o teste é marcado como falhado e o resultado é gravado num ficheiro
\Verb|rejected-test.data| para análise pelo desenvolvedor.

\subsection{Detalhes da implementação}

Este tipo de testagem assume que o resultado da execução é \emph{determinista},
ou seja, para um conjunto de entadas igual o resultado será igual, no entanto
o \Verb|spacecheck.sh| não pode ser considerado determinista pois este depende
de uma variavél externa, o tempo.

O resultado de \Verb|spacecheck.sh| inclui sempre uma data \Verb|YYYYMMDD| de
quando este foi executado, isto é um problema porque o mesmo teste quando
executado em dias diferentes irá apresentar resultados diferentes causando com
que o teste falhe.

% TODO: Add bibliographic reference to libfaketime
Logo para obter determinismo na execução dos testes é necessário correr o
programa sempre com um tempo pré-definido, para tal foi utilizado o programa
\Verb|faketime| do projeto \Verb|libfaketime|, este permite mudar o tempo
reportado a um programa sem ter de mudar o relógio do sistema, sendo assim ideal
para o nosso objetivo de uma ferramenta de fácil utilização para testagem.

\begin{listing}[H]
	\begin{minted}{shell-session}
		$ date +%Y%m%d
		20231104
		$ faketime "2023/11/01" date +%Y%m%d
		20231101
	\end{minted}
	\cprotect\caption{Exemplo do funcionamento de \Verb|faketime|.}
\end{listing}

\section{Resultados}

A implementação de um processo de testagem automática descrito na secção
anterior, permitiu-nos satisfazer os objetivos que delineamos na secção
\ref{sec:testing_metodology}, os testes são todos executados automaticamente (com a
opção de opcionalmente restringir os testes a executar), eles correm em
ambientes que são criados de raiz em cada execução e são declarados num ficheiro
\emph{Bash}.

Mas o mais importante de tudo, foi o facto do envolvimento do desenvolvedor
em todo o processo ter sido cortado significativamente, este agora só necessita
de começar o processo de testagem, verificar os ficheiros de referência
inicialmente e corrigir erros que sejam capturados pela testagem.

Isto traduz-se numa redução do tempo de testagem para \textbf{um teste} de
\numrange{10}{30} segundos em média, para $< 2$ segundos para
\textbf{todos os testes!}

Além da redução do tempo necessário para a testagem, o novo processo foi logo
após a sua implementação capaz de capturar um erro que tinha sido introduzido
anteriormente aquando de mudanças de como os ficheiros eram filtrados. O erro
causava com que o filtro de data de modificação não tivesse qualquer efeito,
este em particular é um dos casos mais difíceis de testar devido ao números de
passos necessários para configurar uma diretoria com datas especificas manualmente.

Concluindo, o novo processo de testagem é mais eficiente e eficaz em garantir
a qualidade do trabalho criado, consideramos assim que o tempo gasto na sua
implementação foi útil e recuperado no longo termo.
