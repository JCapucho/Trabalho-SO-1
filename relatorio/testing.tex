\chapter{Testagem}

\section{Metodologia}

Durante o desenvolvimento do programa \Verb|spacecheck.sh| este foi testado
manualmente com alguns ficheiros e diretórios criados a mão para simular
diferentes casos relevantes ao funcionamento do programa.

No entanto este processo era menos que ideal, pois não só obrigava a gastar
tempo a configurar manualmente os diretórios mas também a verificar que os
resultados corespondiam ao esperado. Mas o maior problema de todos era o erro
humano inerente a realização da testagem desta maneira, pois não só dependemos
do desenvolvedor lembrar-se de todos os casos que têm de testar, mas também que
estes sejam bem configurados de todas as vezes e que os erros que possam
eventualmente aparecer sejam apanhados.

Estes dois últimos pontos em particular são mais propícios de falhar com o
aumento da frequência com que são feitos, o que vai contra o nosso objetivo
de testar frequentemente.

Logo procuramos obter um novo processo de testagem que satisfizesse as seguintes
condições:

\begin{enumdescript}
	\item[Automático]
	todos os testes necessários são executados sem que o desenvolvedor tenha que
	se lembrar de cada um deles individualmente.
	\item[Declarativo]
	todos os testes são configurados uma vez e sempre que os testes são
	executados os diretórios e ficheiros necessários são criados de raíz
	para garantir que os testes são facilmente reproduzidos.
	\item[Transparentes]
	não é necessário a intervenção do desenvolvedor para verificar o resultado
	do teste em casos em que este está correto.
\end{enumdescript}

Como tal desenvolvemos um processo automático de testagem que se baseia no
padrão de \textbf{Golden Testing}. Este consiste em correr um teste com entradas
definidas e constantes para o mesmo teste (no nosso caso estas entradas são os
ficheiros e diretórios) e comparar o resultado com um resultado prévio que se
sabe estar correto.

Esta abordagem é comum no desenvolvimento de \emph{User Interfaces} e em
mensagens de erros de ferramentas pois em ambos destes casos a única propriedade
que faz sentido verificar é o resultado em si e a testagem de componentes
individuais destes sistemas não faz sentido em termos de prevenir erros.
