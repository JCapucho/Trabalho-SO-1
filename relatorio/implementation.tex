\chapter{Implementação}

\section{Spacecheck}

\subsection{Encontrar os diretórios}

Para conseguir gerir e monitorizar o espaço ocupado em disco dentro de
diretórios é primeiro essencial saber quais são os diretórios a ser usados.

Usando os argumentos passados ao script com os diretórios a serem pesquisados,
descobrimos os seus sub-diretórios todos atraves do comando
\mintinline{bash}{find} da seguinte maneira:
\begin{listing}[H]
  \begin{minted}{bash}
    find "$@" -type d -print0 2>/dev/null
  \end{minted}
\end{listing}
Procuramos dentro dos diretórios passados todos os seus sub-diretórios
(\mintinline{bash}{-type d}). Todos os erros não nos são relevantes por isso
não os imprimimos. A forma como os caminhos dos diretórios são impressos também
é relevante sendo que eles acabam com \Verb|'\0'| e não \Verb|'\n'| como normalmente
seriam.

O motivo para fazer a impressão dos caminhos desta maneira será evidente mais
tarde quanto começarmos a iterar sobre eles.

\subsection{Escolher os ficheiros}

\subsection{Calcular o tamanho de cada diretório}

\subsection{Processamento de argumentos}

\subsection{Ordenar o output}

\subsection{Escolher o número de linhas}

\section{Spacerate}

\subsection{Extrair os dados para um dicionário}

\subsection{Obter uma lista de todos os caminhos}

\subsection{Obter as diferenças entre cada ficheiro}

