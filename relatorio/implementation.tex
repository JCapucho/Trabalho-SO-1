\chapter{Implementação}

\section{Introdução}

Neste trabalho foi nos pedido que críassemos dois scripts \emph{Bash}:
\Verb|spacecheck.sh| e \Verb|spacerate.sh|, cada um destes desempenharia uma
função diferente mas complementares.

O script \Verb|spacecheck.sh| é responsável por produzir um relatório que
detalha o espaço ocupado por um diretório e como está distribuído pelos seus
subdiretórios, com opções extra para controlar quais ficheiros contam para esta
métrica.

Os resultados da execução poderiam depois ser comparados com recurso ao script
\Verb|spacerate.sh|, este calcularia a diferença entre os resultados e
apresentaria ao utilizador como o tamanho dos diretórios variou e se foram
introduzidos diretórios novos ou removidos.

Nesta secção vamos detalhar como implementamos os nossos scripts, incluindo não
só como funcionam mas também a lógica por detrás das nossas decisões para os
construir desta maneira.

Os scripts possuem uma parte semelhante aos dois significativa (por exemplo, o
processamento de argumentos e o ordenamento de resultados) como tal iremos
começar por discutir as partes específicas a cada script e de seguida
abordaremos as partes comuns.

\cprotect\section{\Verb|spacecheck.sh|}

O script \Verb|spacecheck.sh| como já foi referido mostra o espaço ocupado por
um diretório e todos os seus sub diretórios, possuindo opções para filtrar os
ficheiros a usar no cálculo do espaço ocupado.

Podemos então dividir a implementação em três grandes problemas a resolver:

\begin{enumerate}
	\item Encontrar os diretórios e subdiretórios
	\item Escolher os ficheiros que contam
	\item Calcular o tamanho do diretório
\end{enumerate}

Vamos então começar por explorar cada um destes problemas e como foram resolvidos.

\subsection{Encontrar os diretórios}\label{sec:implementation_find_dirs}

O primeiro passo então é descobrir todos os diretórios e subdiretórios que terão
de ser considerados pelo script.

O comando \Verb|find| é ideal para esta situação, este permite descobrir
recursivamente os ficheiros e subdiretórios de vários diretórios que são
passados como argumentos. Além disso conseguimos filtrar os resultados para só
obter diretórios. Logo uma primeira versão para resolver este problema seria
parecida com algo semelhante ao código
\ref{code:implementation_first_subdir_find}.

\begin{listing}[H]
	\centering
	\begin{minted}{bash}
    find "$@" -type d
  \end{minted}
	\caption{Exemplo de como obter os subdiretórios}
	\label{code:implementation_first_subdir_find}
\end{listing}

Este código parece ser a solução ideal para o problema, e em alguns aspetos é,
por exemplo, esta primeira versão lida corretamente com argumentos que têm
espaços. Mas o problema desta solução não é o facto de funcionar mas o resultado
que produz.

Se por exemplo, executássemos este comando num diretório \Verb|test| que têm
dois subdiretórios \Verb|a| e \Verb|b|, obteríamos o seguinte resultado:

\begin{listing}[H]
	\centering
	\begin{minted}{text}
    test
    test/a
    test/b
  \end{minted}
	\cprotect\caption{Exemplo do resutlado obtido pelo \Verb|find|}
\end{listing}

Como podemos ver os resultados vêm separados por linhas novas (\Verb|\n|), isto
para a vasta maioria dos casos funciona, no entanto a norma \emph{POSIX} define
que um nome de ficheiro/diretório pode ser constituído por qualquer byte com a
exceção do byte nulo (\bashinline|\0|) e da barra (\bashinline|/|)
\cite[60]{posix}.

Isto implica que \bashinline|a\nd| é um nome completamente válido para um diretório,
se executarmos o mesmo comando no diretório \Verb|test| onde agora existe este
subdiretório, obtemos o seguinte resultado.

\begin{listing}[H]
	\centering
	\begin{minted}{text}
    test/a
    d
    test/a
    test/b
  \end{minted}
	\caption{Resultado obtido com um diretório que contêm uma linha nova}
\end{listing}

Apesar disto ser uma situação rara é uma possibilidade e como tal temos de a
acautelar no nosso script, mas já vimos que o caminho não pode conter bytes
nulos e o \Verb|find| contêm uma opção \Verb|-print0| que invês de separar os
resultados com uma linha nova, vai separar com o byte nulo \cite{find_man}.

Chegamos assim então a segunda iteração do comando para obter os diretórios e
subdiretórios apresentado no código \ref{code:implementation_second_subdir_find}.

\begin{listing}[H]
	\centering
	\begin{minted}{bash}
    find "$@" -type d -print0
  \end{minted}
	\caption{Segundo iteração do comando para obter os subdiretórios}
	\label{code:implementation_second_subdir_find}
\end{listing}

No entanto esta versão ainda tem um problema, como já vimos os diretórios podem
ter nomes pouco convencionais e se o nome começar com um híphen (\Verb|-|),
estes vão ser interpretados pelo \Verb|find| como opções. Para resolver isto
vamos ter de escapar manualmente todos os diretórios que começam com um híphen
\footnote{Mais a frente o mesmo problema terá de ser resolvido noutros comandos,
	estes ou aceitam um argumento especial \Verb|--| que sinaliza o fim das opções,
	ou têm uma opção \Verb|-files0-from|. O primeiro não existe no \Verb|find|
	devido à ordem pouco convencional dos argumentos, o segundo existe em algumas
	versões do \Verb|find|, mas é pouco comum, principalmente em distribuições que
	não são baseadas nas ferramentas GNU. Logo decidimos pela opção manual.}.

\begin{listing}[H]
	\centering
	\begin{minted}{bash}
    DIRS_TO_SEARCH=()

    for dir in "$@"; do
      if [ ! -d "$dir" ]; then
        1>&2 echo "ERROR: \"$dir\" does not exist"
      fi
      case "$dir" in
        -*)
          if [ ! -d "$dir" ]; then
            1>&2 echo "+note: did you mean to pass an argument?"
            1>&2 echo "       all arguments must precede the directory paths"
          fi
          # Add `./` to directories beginning with a dash (`-`), so that find
          # doesn't mistake them, for options.
          DIRS_TO_SEARCH+=("./$dir");;
        *)
          DIRS_TO_SEARCH+=("$dir");;
      esac
    done
  \end{minted}
	\caption{Implementação do escape de diretórios começadas por híphen}
	\label{code:implementation_escape_options_find}
\end{listing}

O código \ref{code:implementation_escape_options_find}, mostra a implementação
final para escapar os híphens no início do nome dos diretórios. Este código
ainda implementa mensagens de erro para diretórios que não existem e mensagens
de ajuda casos esses diretórios pareçam ser opções.

Finalmente, é apresentado no código \ref{code:implementation_final_subdir_find}
a implementação final do comando para obter os diretórios e subdiretórios.

\begin{listing}[H]
	\centering
	\begin{minted}{bash}
    find "${DIRS_TO_SEARCH[@]}" -type d -print0
  \end{minted}
	\caption{Iteração final do comando para obter os subdiretórios}
	\label{code:implementation_final_subdir_find}
\end{listing}

\subsection{Escolher os ficheiros}\label{sec:implementation_choosing_files}

Agora que já temos os diretórios é necessário selecionar todos os ficheiros que
pertencem a um dado diretório e passam nos filtros definidos pelo utilizador.

Assumindo que o diretório a procurar se encontra na variável \Verb|path|, podemos
adaptar o comando utilizado na secção anterior para procurar todos os ficheiros
num dado diretório.

\begin{listing}[H]
	\centering
	\begin{minted}{bash}
    find "$path" -type f -print0
  \end{minted}
	\caption{Comando para listar todos os ficheiros num diretório}
\end{listing}

No entanto, isto não é suficiente pois é necessário filtrar os ficheiros de
acordo com as opções indicadas pelo utilizador. Para tal é introduzido um array
novo \Verb|FIND_OPTS|, este irá conter opções a passar ao \Verb|find| de acordo
com as opções especificadas pelo utilizador para o \Verb|spacecheck.sh|.

Logo o comando final para obter apenas os ficheiros que satisfazem os filtros
será o seguinte.

\begin{listing}[H]
	\centering
	\begin{minted}{bash}
    find "$path" "${FIND_OPTS[@]}" -type f -print0
  \end{minted}
	\caption{Comando para filtrar os ficheiros num diretório}
\end{listing}

A maneira como são processados os argumentos é detalhada na secção
\ref{sec:implementation_arg_parsing}. Mas vamos analisar como as opções do
\Verb|spacecheck.sh| são convertidas nas opções do \Verb|find|.

O valor passado as opções está na variável \Verb|OPTARG|, mais uma vez isto será
falado em maior detalhe na secção de processamento de argumentos.

\subsubsection{Filtragem do nome de ficheiro}

Vamos começar por explorar como o script lida com a filtragem do nome de
ficheiro utilizando \emph{regex}.

\begin{listing}[H]
	\centering
	\begin{minted}{bash}
    pattern="$OPTARG"
    FIND_OPTS+=("-regextype" "posix-extended" "-regex" "^.*/$pattern$")
  \end{minted}
	\cprotect\caption{Construção do filtro de \emph{regex} sobre o nome do ficheiro}
\end{listing}

Começamos por escolher o tipo de expressão regular que queremos usar, neste
caso \Verb|posix-extended|, isto é feito por uma questão de usabilidade pois o
tipo de \emph{regex} por defeito é o \Verb|findutils-default| que utiliza uma
sintaxe muito pouco convencional e não têm alguns dos operadores comuns
\cite{find_man} como a estrela (\Verb|*|) que espera zero ou mais ocorrências
da expressão.

De seguida é passada a expressão regular, no entanto esta não pode ser passada
diretamente e precisa de algum código extra a sua volta. Isto é necessário pois
a opção \Verb|-regex| não aplica o \emph{regex} ao nome do ficheiro mas sim ao
caminho todo. Como tal começamos por ancorar o \emph{regex} ao início do caminho
(\Verb|^|) e consumimos todos os caracteres (\Verb|.*|) até a última barra
\footnote{O \emph{regex} consome até a última barra porque é \emph{greedy}, ou
	seja, este não vai parar na primeira barra que encontrar, mas vai continuar a
	tentar consumir o máximo possível desde que respeite o padrão.}
, a este ponto o \emph{regex} está no início do nome do ficheiro por isso
aplicamos o \emph{regex} do utilizador, finalmente ancoramos o \emph{regex} ao
fim (\Verb|$|) isto garante que o nome do ficheiro passa totalmente no
\emph{regex} do utilizador e não apenas parcialmente (por exemplo, se o
utilizador passar o \emph{regex} \Verb|abc| nós queremos apenas aceitar
ficheiros com o nome \Verb|abc|, mas se o regex não tivesse ancorado também
aceitaríamos \Verb|abcd| entre outros).

\subsubsection{Filtragem por data de modificação}

Outra opção a disposição do utilizador, é a filtragem por data de modificação,
esta encontra-se implementada da seguinte maneira.

\begin{listing}[H]
	\centering
	\begin{minted}{bash}
			data="$OPTARG"
			FIND_OPTS+=('!' "-newermt" "$data")
  \end{minted}
	\caption{Construção do filtro por data de modificação}
\end{listing}

A opção \Verb|-newermt| faz com que o \Verb|find| apenas procure por ficheiros
com data de modificação mais recente à data passada, nós queremos o contrário por
isso utilizamos a opção \Verb|!| que inverte o filtro que aparece a sua direita.

As datas são processadas do mesmo modo que são processadas pelo script
\Verb|date| \cite{find_man}, logo não é necessário qualquer passo extra para
converter da descrição textual passada pelo utilizador para tempo máquina.

\subsubsection{Filtragem por tamanho de ficheiro}

O último filtro implementado permite ao utilizador definir um tamanho mínimo que
os ficheiros devem ter para serem contabilizados. A nossa implementação permite
além disto filtrar ficheiros que tenham menos que o tamanho passado e mais (mas
não exatamente igual) através do uso dos prefixos \Verb|-| e \Verb|+|
respetivamente.

\begin{listing}[H]
	\centering
	\begin{minted}{bash}
    tamanho="$OPTARG"
    if [[ "${tamanho:0-1}" =~ ^[0-9]+$ ]]; then
      tamanho+="c"
    fi
    if [[ "${tamanho:0:1}" =~ ^[0-9]+$ ]]; then
      FIND_OPTS+=("(" "-size" "$tamanho" "-o" "-size" "+$tamanho" ")")
    else
      FIND_OPTS+=("-size" "$tamanho")
    fi
  \end{minted}
	\caption{Construção do filtro por tamanho do ficheiro}
\end{listing}

O primeiro \Verb|if| verifica se o último caráter é um número e caso seja
adiciona um sufixo \Verb|c| ao tamanho, isto serve para o \Verb|find|
interpretar o tamanho em bytes em vez de blocos, como é feito por omissão pelo
\Verb|find|. Este sufixo condicional permite ao utilizador escolher outra
unidade se desejar, bastando adicionar o sufixo correto.

No segundo \Verb|if| verificamos se o primeiro caráter é um número, se for o
script vai definir um filtro composto no \Verb|find| este é delineado por
parênteses, caso contrário o filtro \emph{ou} (\Verb|-o|) não se aplicaria só ao
segundo filtro mas ao resto dos filtros todos. O primeiro filtro \Verb|-size|
não tem nenhum prefixo logo o \Verb|find| procura por ficheiros com exatamente o
tamanho especificado, este é seguido de um \emph{ou} como já foi referido que é
complementado com outro filtro \Verb|-size| mas este utiliza um tamanho com o
prefixo \Verb|+| que apenas aceita ficheiros estritamente maiores que o tamanho.
Esta combinação é necessária para obter todos os ficheiros maiores ou iguais ao
tamanho mínimo.

Se o primeiro caráter não for um número, o script considera que o tamanho têm um
prefixo e passa diretamente ao \Verb|find|, podendo assim o utilizador procurar
por ficheiros maiores e menores (mas não iguais) ao tamanho com o prefixo
\Verb|+| e \Verb|-| respetivamente.

\subsection{Calcular o tamanho de cada diretório}
\label{sec:implementation_calculate_size}

Agora que já temos uma lista de todos os ficheiros pertencentes a um
diretório, falta-nos apenas calcular o tamanho de todos ficheiros.
Para este fim vamos utilizar o comando \bashinline{du} que permite calcular o
espaço ocupado de uma lista de ficheiros.

A nossa lista de ficheiros vêm separada por bytes nulos, logo invés de passar os
ficheiros diretamente, vamos ter de utilzar a opção \Verb|--files0-from|, esta
causa com que o \bashinline{du}, invés de ler a lista de ficheiros do
\emph{stdin} separada por linhas novas, vá ler a lista de ficheiros separada por
bytes nulos do ficheiro passado como valor da opção \cite{du_man}, nós utilizamos
o valor especial \Verb|-| que sinaliza que o ficheiro p \emph{stdin}, desta
maneira conseguimos ler do \emph{stdin} separado por bytes nulos.

O comando \bashinline{du} por defeito mostra o espaço ocupado por cada ficheiro
mas não um total, no entanto, se passarmos a opção \Verb|-c| este irá imprimir
na última linha um total correspondente a soma do tamanho de todos os ficheiros
\cite{du_man}.

A primeira versão do comando que foi utilizado é apresentada no código
\ref{code:implementation_first_size_calcultate}.

\begin{listing}[H]
	\centering
	\begin{minted}{bash}
    du -c --files0-from=- | tail -n1 | cut -f1
  \end{minted}
	\caption{Primeira iteração do comando para calcular o tamanho dos ficheiros}
	\label{code:implementation_first_size_calcultate}
\end{listing}

O comando baseia-se no \bashinline{du} com as opções discutidas em cima, o
resultado da sua execução é depois redirecionado para o \bashinline{tail} que
irá filtrar tudo com a exceção da última linha (\Verb|-n1|), que será mais uma
vez redirecionada desta vez para o \bashinline{cut} que irá retirar a primeira
coluna (\Verb|-f1|), esta corresponde ao valor do tamanho total.

Esta versão tinha apenas o problema que o \bashinline{du} por omissão conta o
número de blocos ocupados no disco pelo ficheiro, no entanto nós queremos o
número de bytes ocupados pelo conteúdo do ficheiro (e não o numero de bytes do
bloco no disco onde reside o ficheiro). Isto pode ser resolvido com a opção
\Verb|-b| que não só retorna o resultado em bytes mas também conta o tamanho do
conteúdo invés dos blocos ocupados pelo ficheiro \cite{du_man}.

\begin{listing}[H]
	\centering
	\begin{minted}{bash}
    du -cb --files0-from=- | tail -n1 | cut -f1
  \end{minted}
	\caption{Iteração final do comando para calcular o tamanho dos ficheiros}
	\label{code:implementation_final_size_calcultate}
\end{listing}

A versão do código \ref{code:implementation_final_size_calcultate} é a que foi
implementada no script final, esta produz o total do conteúdo dos ficheiros em
bytes.

\subsection{Integração final}\label{sec:implementation_spacecheck_integration}

Agora que já foram resolvidos todos os problemas individualmente resta apenas
integrá-los todos de maneira a conseguir produzir a lista de tamanho ocupado por
diretório.

O primeiro passo é integrar o código para encontrar diretórios desenvolvido na
secção \ref{sec:implementation_find_dirs} de modo a conseguir processar cada
diretório individualmente. Isto é feito com recurso a um loop
\bashinline{while read}.

\begin{listing}[H]
	\centering
	\begin{minted}{bash}
    find "${DIRS_TO_SEARCH[@]}" -type d -print0 2>/dev/null | \
    while IFS= read -r -d $'\0' path; do
      ...
    done
  \end{minted}
	\caption{Iteração sobre os diretórios encontrados}
\end{listing}

Os erros na procura dos diretórios não nos são relevantes, por isso o
\emph{stderr} (\emph{file descriptor} 2) é redirecionado para o
\Verb|/dev/null|.

O loop em si começa por atribuir a variável \bashinline{IFS} o valor nulo, isto
é necessário para garantir que o \bashinline{read} não separa por espaços o
resultado nem consome espaços no final do resultado, os diretórios podem
contê-los e nós queremos preservá-los.

O \bashinline|-r| é necessário caso contrário o \bashinline{read} interpretaria
as barras invertidas como \emph{escape sequences} \cite[3191-3193]{posix}, os
ficheiros podem conter barras invertidas e nós queremos o seu valor literal logo
precisamos de desativar este processamento.

A opção \bashinline|-d $'\0'| também têm um papel importante, visto que como
a lista de diretórios se encontra separada por bytes nulos (\Verb|'\0'|) é
essencial mudar o delimitador do \bashinline{read} (que por omissão é a linha
nova) para o byte nulo, a string é precedida de um dollar (\Verb|$|) pois esta
causa com que a \emph{Bash} processe a string com \emph{escape sequences}, caso
contrário o valor literal de barra invertida seguida de zero seria passado ao
\bashinline{read}.

De seguida é necessário juntar os comandos para escolher os ficheiros
(desenvolvido na secção \ref{sec:implementation_choosing_files}) e calcular
o tamanho (desenvolvido na secção \ref{sec:implementation_calculate_size}). O
resultado do primeiro pode ser redirecionado diretamente para o segundo visto
que este já foi adaptado para receber a lista de ficheiros separada por bytes
nulos.

\begin{listing}[H]
	\centering
	\begin{minted}{bash}
    size=$(find "$path" "${FIND_OPTS[@]}" -type f -print0  2>/dev/null | \
         du -cb --files0-from=- 2>/dev/null | \
         tail -n1 | cut -f1)
  \end{minted}
	\caption{Cálculo do tamanho total de um diretório}
\end{listing}

Mais uma vez as mensagens de erro são ignoradas, no entanto desta vez os erros
não são simplesmente ignorados, pois se algum tiver ocorrido (por exemplo,
devido a falta de permissões) o tamanho é suposto aparecer como \Verb|NA|.

\begin{listing}[H]
	\centering
	\begin{minted}{bash}
    size=$(find "$path" "${FIND_OPTS[@]}" -type f -print0  2>/dev/null | \
         du -cb --files0-from=- 2>/dev/null | \
         tail -n1 | cut -f1)

    if [ "$?" -ne 0 ]; then
      size="NA"
    fi
  \end{minted}
	\caption{Cálculo do tamanho total de um diretório (com deteção de erros)}
	\label{code:implementation_dir_size_NA}
\end{listing}

No código \ref{code:implementation_dir_size_NA} o código de retorno é
verificado e se este não for 0 (ocorreu um erro) o tamanho é substituído por
\Verb|NA|.

Normalmente a variável \Verb|$?| apenas contêm o código de erro do último
comando executado, no entanto nós queremos o código de erro da \emph{pipeline}
toda, isto poderia ser feito verificando os códigos de erro no array
\Verb|PIPESTATUS|, no entanto isto requer escrever uma condição para cada
comando na \emph{pipeline} o que é menos que ideal.

Mas a \emph{Bash} pode ser configurada para devolver o código de erro da
\emph{pipeline} toda diretamente na variável \Verb|$?|, isto é feito através da
opção \Verb|pipefail| \cite{bash_man}, esta pode ser ativada como é apresentado
no código \ref{code:implementation_pipefail}.

\begin{listing}[H]
	\centering
	\begin{minted}{bash}
    set -o pipefail
  \end{minted}
	\cprotect\caption{Ativação da opção \Verb|pipefail|}
	\label{code:implementation_pipefail}
\end{listing}

Finalmente, basta apenas juntar estas duas partes que produzimos até agora.

\begin{listing}[H]
	\centering
	\begin{minted}{bash}
    set -o pipefail

    find "${DIRS_TO_SEARCH[@]}" -type d -print0 2>/dev/null | \
    while IFS= read -r -d $'\0' path; do
      size=$(find "$path" "${FIND_OPTS[@]}" -type f -print0  2>/dev/null | \
           du -cb --files0-from=- 2>/dev/null | \
           tail -n1 | cut -f1)
      if [ "$?" -ne 0 ]; then
        size="NA"
      fi

      echo -e "$size\t$path"
    done
  \end{minted}
	\cprotect\caption{Implementação do algoritmo principal do \Verb|spacecheck.sh|}
\end{listing}

\cprotect\section{\Verb|spacerate.sh|}

O script \Verb|spacerate.sh| após fornecido dois ficheiros com resultados da
execução do \Verb|spacecheck.sh| calcula a diferença, indicando também se os
diretórios foram criados ou removidos de uma execução para a outra.

A implementação pode então ser dividida duas partes:

\begin{enumerate}
	\item Extrair os diretórios
	\item Comparar os diretórios
\end{enumerate}

\subsection{Extrair os diretórios}

O primeiro e principal passo para começar a comparar os diretórios é descobrir
que diretórios existem e colocá-los numa estrutura de dados que seja fácil de
manipular. Para isso começamos por declarar dois dicionários, um para cada
ficheiro, estes vão conter um mapeamento do diretório para o seu tamanho.
Além destes dois, um terceiro dicionário será criado que será usado para
contabilizar todos os diretórios encontrados a partir dos dois ficheiros,
sem repetições.

\begin{listing}[H]
	\centering
	\begin{minted}{bash}
    declare -A SPACECHECK_NEW
    declare -A SPACECHECK_OLD

    declare -A DIRECTORIES
  \end{minted}
	\cprotect\caption{Declaração dos dicionários usados no \Verb|spacerate.sh|}
\end{listing}

Para facilitar a leitura do código, o caminho para o primeiro ficheiro será
guardado na variável \Verb|NEWFILE| e o do segundo na variável
\Verb|OLDFILE|.

Os ficheiros vão ter de ser iterados linha a linha para poderem ser extraídos os
dados necessários para popular os dicionários. Para este fim foi utilizada um
loop \bashinline{while read} como é apresentado no código
\ref{code:implementation_file_loop}.

\begin{listing}[H]
	\centering
	\begin{minted}{bash}
    while IFS= read -r -d $'\n' line; do
      ...
    done < <(tail -n +2 -- "$NEWFILE")
  \end{minted}
	\cprotect\caption{Iteração sobre o ficheiro do \Verb|spacecheck.sh|}
	\label{code:implementation_file_loop}
\end{listing}

Este loop é estruturado de uma maneira semelhante ao que foi utilizado na secção
\ref{sec:implementation_spacecheck_integration}, mas invés de bytes nulos o
delimitador escolhido é o caráter de linha nova.

É de notar ainda que este loop utiliza \emph{Process Substitution}
\cite{bash_man}, invés de uma \emph{pipe} como é normal. Isto é necessário pois
um loop \bashinline{while} precedido por uma \emph{pipe} vai ser executado numa
\emph{shell} nova e não terá acesso as variáveis enquanto que o redirecionamento
do resultado do \emph{Process Substitution} será executado na mesma
\emph{shell}.

O \bashinline{tail -n +2} é utilizado para ignorar a primeira linha do ficheiro,
que contém o cabeçalho.

Este loop vai ser feito não só para o \Verb|NEWFILE| mas também para
\Verb|OLDFILE|.

Dentro de cada um destes loops, cada linha é processada do seguinte modo:

\begin{listing}[H]
	\centering
	\begin{minted}{bash}
      size=$(echo "$line" | cut -f1)
      name=$(echo "$line" | cut -f2-)
      SPACECHECK_NEW["$name"]="$size"
      DIRECTORIES["$name"]=1
  \end{minted}
	\cprotect\caption{Processamento de uma linha do resultado do \Verb|spacecheck.sh|}
\end{listing}

O \bashinline{cut -f1} é utilizado para obter a primeira coluna e o
\bashinline{cut -f2-} para obter a segunda coluna e todas as que se apresentam a
seguir a ela (apesar do resultado do \Verb|spacecheck.sh| ser projetado apenas
com duas colunas, devido ao facto que o nome dos diretórios podem conter
espaços, outras \enquote{colunas} podem ser formadas).

Após isso é adicionado ao dicionário \Verb|SPACECHECK_NEW| o nome do ficheiro
como chave e o tamanho como valor. No loop dedicado ao ficheiro
\Verb|OLDFILE| o dicionário atualizado é o \Verb|SPACECHECK_OLD|.

Ao mesmo tempo também adicionamos ao dicionário \Verb|DIRECTORIES| o nome do
ficheiro como chave e o valor 1, que como é um dicionário não repete as chaves,
logo vamos obter um conjunto com todos os diretórios dos dois ficheiros.

\subsection{Comparar os diretórios}

Agora que já temos os dados guardados em dicionários podemos começar a comparar
os dois ficheiros. Como tal vamos começar por iterar sobre os diretórios.

Como foi referido na secção anterior, todos os diretórios que foram encontrados
em ambos os ficheiros foram guardados no dicionário \Verb|DIRECTORIES|, logo
basta iterar sobre as chaves do dicionário, isto é alcançado ao prefixar o
dicionário com um ponto de exclamação (\Verb|!|) como é demonstrado no código
\ref{code:implementation_dirs_loop}.

\begin{listing}[H]
	\centering
	\begin{minted}{bash}
    for key in "${!DIRECTORIES[@]}"; do
      # ...
    done
  \end{minted}
	\caption{Iteração sobre os diretórios encontrados}
	\label{code:implementation_dirs_loop}
\end{listing}

De seguida, o tamanho do diretório é consultado nos dicionários que foram criados
e populados na secção anterior, de modo a simplificar o código estes valores são
guardados em variáveis.

\begin{listing}[H]
	\centering
	\begin{minted}{bash}
      old_size="${SPACECHECK_OLD[$key]}"
      new_size="${SPACECHECK_NEW[$key]}"
  \end{minted}
	\caption{Consulta do tamanho do diretório em ambas as versões}
\end{listing}

Agora é necessário verificar os casos em que o diretório está apenas presente
num dos ficheiros. Para tal começamos por verificar se o \Verb|old_size| está
vazio, isto indicará que o diretório não existe no ficheiro antigo mas existe
no novo, logo a diferença não precisa de ser calculada e o ficheiro é marcado
como \Verb|NEW| no fim da linha.

\begin{listing}[H]
	\centering
	\begin{minted}{bash}
      if [ -z "$old_size" ]; then
        echo -e "$new_size\t$key\tNEW"
      ...
  \end{minted}
	\caption{Caso em que o diretório não existe no ficheiro antigo}
\end{listing}

Caso o \Verb|new_size| esteja vazio o diretório não existe no ficheiro novo,
mas existe no antigo, logo não precisamos de calcular a diferença, sendo apenas
adicionado um \Verb|-| no tamanho para denotar que este foi removido, além disso
é adicionado ao fim da linha a string \Verb|REMOVED|.

\begin{listing}[H]
	\centering
	\begin{minted}{bash}
    ...
    elif [ -z "$new_size" ]; then
      if [ "$old_size" == "0" ] || [ "$old_size" == "NA" ]; then
        display_size="$old_size"
      else
        display_size="-$old_size"
      fi

      echo -e "$display_size\t$key\tREMOVED"
    ...
  \end{minted}
	\caption{Caso em que o diretório não existe no ficheiro novo}
	\label{code:implementation_new_size_empty}
\end{listing}

É de notar que o código \ref{code:implementation_new_size_empty} é um pouco mais
complexo porque lida com dois casos extra para produzir resultados melhores para
o utilizador. Estes são quando o tamanho no ficheiro velho é \Verb|0| ou
\Verb|NA|, nestas situações não faz sentido adicionar o prefixo \Verb|-| ao
tamanho.

Se o tamanho existir em ambos os ficheiros existe mais um caso especial que
precisa de ser verificado, este é se alguns dos tamanhos é \Verb|NA|, nestes
casos a diferença não é calculada e o \Verb|NA| é propagado.

\begin{listing}[H]
	\centering
	\begin{minted}{bash}
    ...
    elif [ "$old_size" == "NA" ] || [ "$new_size" == "NA" ]; then
      echo -e "NA\t$key"
    ...
  \end{minted}
	\caption{Propagação do NA}
\end{listing}

Por fim, se os dois tamanhos existirem e não forem \Verb|NA| então calculamos a
diferença do diretório nos dois ficheiros e resultado é impresso.

\begin{listing}[H]
	\centering
	\begin{minted}{bash}
    else
      diff=$((new_size - old_size))
      echo -e "$diff\t$key"
    fi
    ...
  \end{minted}
	\caption{Cálculo da diferença entre os dois ficheiros}
\end{listing}

\section{Processamento de argumentos}\label{sec:implementation_arg_parsing}

O processamento dos argumentos em ambos os scripts é delegado ao comando
\bashinline{getopts}, este recebe um padrão, que especifica os argumentos que são
aceites e se têm valor ou não, e a variável onde o argumento que está a ser
processado atualmente vai ser guardado.

No código \ref{code:implementation_spacecheck_args} é apresentado um excerto de
como o \bashinline{getopts} é utilizado no \Verb|spacecheck.sh| para processar
os seus argumentos.

\begin{listing}[H]
	\centering
	\begin{minted}{bash}
    while getopts ":hran:d:s:l:" o; do
        case "${o}" in
        h)
          help
          ;;
        ...
        : )
          help "Missing argument for \`-$OPTARG\`"
          ;;
        ? )
          help "Unknown option \`-$OPTARG\`"
          ;;
        * )
          help "Unknown option \`-$o\`"
          ;;
        esac
    done
  \end{minted}
	\cprotect\caption{Processamento de argumentos no \Verb|spacecheck.sh|}
	\label{code:implementation_spacecheck_args}
\end{listing}

O padrão começa com o símbolo de dois pontos (\Verb|:|), isto indica ao
\bashinline{getopts} para não emitir mensagens \cite[2837]{posix}, nós fazemos
isto para ter um controlo maior sobre os erros que são emitidos, os nossos erros
são acompanhados de um menu de ajuda como é apresentado no código
\ref{code:implementation_spacecheck_help}.

\begin{listing}[H]
	\centering
	\begin{minted}{text}
    Usage: ./spacecheck.sh [options] [dir...]

    Options:
      -h:          Shows this message
      -n PATTERN:  Filter files by name according to the pattern
      -d N:        Filter files by date of modification
      -s N:        Filter files by file size
      -r:          Print in reverse order
      -a:          Order by file name
      -l N:        Only show up to N lines
  \end{minted}
	\cprotect\caption{Menu de ajuda do \Verb|spacecheck.sh|}
	\label{code:implementation_spacecheck_help}
\end{listing}

Este menu também pode ser acessado através da opção \Verb|-h|.

As opções são definidas adicionando o seu caráter a string padrão, se a opção
precisar de um argumento está tem que ser sufixada com o símbolo de dois pontos
(\Verb|:|), o argumento depois estará disponível na variável \Verb|OPTARG|
\cite[2838]{posix}.

Depois apenas é necessário código no case para lidar com a opção.

\begin{listing}[H]
	\centering
	\begin{minted}{bash}
    while getopts ":hran:d:s:l:" o; do
        case "${o}" in
        ...
        l)
          linhas="$OPTARG"
          if [ -z "$linhas" ]; then
            help "Missing argument for \`-$o\`"
          fi
          HEAD_OPTS+=("-n" "$linhas")
          ;;
        ...
        esac
    done
  \end{minted}
	\caption{Exemplo da implementação de uma opção}
\end{listing}

\section{Ordenação do resultado}

Em ambos os scripts, a maneira como o resultado é ordenado pode ser controlada
de duas maneira, primeiro o utilizador pode escolher se quer os resultados
ordenados por tamanho (usado por omissão) ou por nome do diretório. Em segundo
lugar o utilizador pode escolher ordenar por ordem inversa.

Estas escolhas são guardadas em duas variáveis \Verb|REVERSE_SORT| e
\Verb|NAME_SORT|, que por defeito são \bashinline{false}.

\begin{listing}[H]
	\centering
	\begin{minted}{bash}
    NAME_SORT=false
    REVERSE_SORT=false
  \end{minted}
	\caption{Variavéis responsáveis pelo ordenamento do resultado}
\end{listing}

As opções \Verb|-r| e \Verb|-a| a disposição do utilizador mudam o valor de
\Verb|REVERSE_SORT| e \Verb|NAME_SORT| respetivamente para \bashinline{true}.

\begin{listing}[H]
	\centering
	\begin{minted}{bash}
    ...
      r)
        REVERSE_SORT=true
        ;;
      a)
        NAME_SORT=true
        ;;
    ...
  \end{minted}
	\cprotect\caption{Processamento das opções \Verb|-r| e \Verb|-a|}
\end{listing}

O código responsável por controlar o ordenamento começa por calcular se é
necessário inverter a ordem, isto no caso da ordenação por nome do diretório
corresponde diretamente a presença da opção \Verb|-r|. No entanto, a ordenação
por tamanho é implementada tal que a ordem decrescente é usada por omissão,
mas o \bashinline{sort} utiliza a noção contrária (por omissão ordena por ordem
crescente), logo a ordem inversa se a opção \Verb|-r| \textbf{não} estiver
presente.

\begin{listing}[H]
	\centering
	\begin{minted}{bash}
    NAME_REVERSE_SUFFIX=$([ "$REVERSE_SORT" = false ] && echo "" || echo "r")
    SIZE_REVERSE_SUFFIX=$([ "$REVERSE_SORT" = true ] && echo "" || echo "r")
  \end{minted}
	\caption{Processamento da inversão da ordem de ordenamento}
	\label{code:implementation_reverse_sort}
\end{listing}

Como pode ser observado no código \ref{code:implementation_reverse_sort}, a
decisão de inverter ou não a ordem é guardada através da presença do caráter
\Verb|r| ou não na variável. Isto é justificado se observarmos o código
\ref{code:implementation_sort_fields} que constrói os argumentos que serão
passados ao \bashinline{sort}

\begin{listing}[H]
	\centering
	\begin{minted}{bash}
    SORT_BY_NAME=("-k" "2$NAME_REVERSE_SUFFIX")
    SORT_BY_SIZE=("-k" "1,1n$SIZE_REVERSE_SUFFIX")
  \end{minted}
	\caption{Construção das opções de ordenamento}
	\label{code:implementation_sort_fields}
\end{listing}

A opção \Verb|-k| do comando \bashinline{sort} é usada para escolher as colunas
onde este vai operar, no caso do ordenamento por tamanho \Verb|1,1| ou seja apenas
a primeira coluna e \Verb|2| para o nome do diretório que irá ordenar com todas
as colunas da segunda para a frente (como já foi explicado anteriormente a
presença de espaços no nome do diretório pode levar a criação de mais
\enquote{colunas}).

Mas também é aqui definido como devem ser processadas as linhas, além do sufixo
\Verb|r|, adicionado se a ordem for invertida, o tamanho também têm o prefixo
\Verb|n| que indica ao \bashinline{sort} que a coluna deve ser interpretada
tendo em atenção o seu valor numérico \cite[3248]{posix}.

Finalmente o array \Verb|SORT_OPTS| é produzido que vai conter a ordem com que
estas opções devem ser aplicadas.

\begin{listing}[H]
	\centering
	\begin{minted}{bash}
    if [ "$NAME_SORT" = true ] ; then
      SORT_OPTS+=( "${SORT_BY_NAME[@]}" "${SORT_BY_SIZE[@]}" )
    else
      SORT_OPTS+=( "${SORT_BY_SIZE[@]}" "${SORT_BY_NAME[@]}" )
    fi
  \end{minted}
	\caption{Seleção das opções de ordenamento}
\end{listing}

Foi feita a escolha para ordenar com base em ambos os critérios, mudando apenas
a ordem, pois desta maneira os resultados são mais estáveis de execução para
execução. Caso contrário seria completamente possível dadas as mesma condições
obter ordens diferentes quando dois diretórios tivessem o mesmo tamanho.

Finalmente, o comando \bashinline{sort} é adicionado em série, a seguir a parte
principal do script, através de uma \emph{pipe} (\Verb_|_).

\begin{listing}[H]
	\centering
	\begin{minted}{bash}
    find "${DIRS_TO_SEARCH[@]}" -type d -print0 2>/dev/null | \
    while IFS= read -r -d $'\0' path; do
      ...
    done | \
    sort "${SORT_OPTS[@]}"
  \end{minted}
	\caption{Exemplo da integração do passo de sorteamento nos scripts}
\end{listing}

\section{Escolher o número de linhas}

O utilizador pode ainda em ambos os scripts controlar o número de linhas que
devem ser impressas através da opção \Verb|-l|.

Esta adiciona a um array \Verb|HEAD_OPTS| à opção \Verb|-n| com o número de
linhas que o utilizador especificou, isto indica ao comando \bashinline|head|
quantas linhas deve imprimir.

\begin{listing}[H]
	\centering
	\begin{minted}{bash}
    ...
      l)
        linhas="$OPTARG"
        HEAD_OPTS+=("-n" "$linhas")
    ...
  \end{minted}
	\cprotect\caption{Processamento da opção \Verb|-l|}
\end{listing}

Mais uma vez isto é integrado no programa através do uso de uma \emph{pipe}
(\Verb_|_), mas desta vez é utilizada uma condicional para decidir se deve ser
usado o comando \bashinline|cat|, este passa o resultado sem o manipular,
ou o comando \bashinline|head| com as opções definidas em cima. Deste modo é
possível aplicar o \bashinline|head| apenas se o resultado tiver de ser limitado.

\begin{listing}[H]
	\centering
	\begin{minted}{bash}
    find "${DIRS_TO_SEARCH[@]}" -type d -print0 2>/dev/null | \
    while IFS= read -r -d $'\0' path; do
      ...
    done | \
    sort "${SORT_OPTS[@]}" \
    [ "${#HEAD_OPTS[@]}" -lt 1 ] && cat || head "${HEAD_OPTS[@]}"
  \end{minted}
	\caption{Exemplo da integração do passo de escolhas de linhas nos scripts}
\end{listing}
